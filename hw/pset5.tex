\documentclass[12pt,letterpaper,cm]{hmcpset}
\usepackage[margin=1in]{geometry}
\usepackage{graphicx}
\usepackage{amsmath,amssymb}
\usepackage{algorithm2e}
\usepackage{enumerate}

% info for header block in upper right hand corner
\name{\_\_\_\_\_\_\_\_}
\class{Math 389L}
\assignment{Problem Set 5}
\duedate{Friday, March 29, 2019}
\setlength\parindent{0pt}

\newcommand\x{\boldsymbol{x}}
\newcommand\y{\boldsymbol{y}}
\renewcommand\b{\boldsymbol{b}}
\renewcommand\a{\boldsymbol{a}}
\newcommand\A{\boldsymbol{A}}
\renewcommand\S{\boldsymbol{S}}
\newcommand\E{\mathbb{E}}
\renewcommand\P{\mathbb{P}}

\begin{document}

\begin{problem}
    Let $\x_1,\x_2,\ldots,\x_n$ be independent Rademacher random variables
    (that is, $\x_i = \pm 1$ with equal probability) and $\a=(a_1,a_2,\ldots,a_n)$
    be a sequence of real numbers.
\begin{enumerate}[(a)]
    \item Show that $\E e^{s \x_i} = \cosh(s)$.
    \item Prove that $\cosh(s) \leq e^{s^2/2}$.
    \item Use (a), (b), and Markov's inequality to prove that $$\P\biggl(\biggl|\sum_{i=1}^n a_i \x_i\biggr| \geq \epsilon\biggr) \leq 2e^{-\tfrac{\epsilon^2}{2\|\a\|_2^2}}.$$ \textit{Hint:} Bound each tail $\P\bigl(\sum a_i \x_i \geq \epsilon\bigr)$ and $\P\bigl(\sum a_i \x_i \leq -\epsilon\bigr)$ separately and use a union bound. Try to use symmetry where possible.
    \item Conclude that the sum $\sum_{n=1}^\infty \frac{\x_n}{n}$ converges with probability 1 (i.e. almost surely). Remark why this is interesting.
\end{enumerate}
\end{problem}

\begin{solution}
    \vfill
\end{solution}

\end{document}
