\documentclass[12pt,letterpaper,cm]{hmcpset}
\usepackage[margin=1in]{geometry}
\usepackage{graphicx}
\usepackage{amsmath,amssymb}
\usepackage{algorithm2e}
\usepackage{enumerate}

% info for header block in upper right hand corner
\name{\_\_\_\_\_\_\_\_}
\class{Math 389L}
\assignment{Problem Set 6}
\duedate{Tuesday, April 2, 2019}
\setlength\parindent{0pt}

\newcommand\x{\boldsymbol{x}}
\newcommand\y{\boldsymbol{y}}
\renewcommand\b{\boldsymbol{b}}
\renewcommand\a{\boldsymbol{a}}
\newcommand\A{\boldsymbol{A}}
\renewcommand\S{\boldsymbol{S}}
\newcommand\E{\mathbb{E}}
\renewcommand\P{\mathbb{P}}

\begin{document}

\textbf{Note:} You'll want to read Chapter 2 of Vershynin, especially Section 2.4, before answering these problems.

\begin{problem}[Vershynin 2.4.2]
    Consider the random graph $G \sim G(n,p)$ with expected degrees $d\leq C \log n$ for some constant $C \geq 1$. Show that with high probability (say 0.9), all the vertices of $G$ have degree $\mathcal{O}(\log n)$.
\end{problem}

\begin{solution}
    \vfill
\end{solution}

\begin{problem}[Vershynin 2.4.3]
    Consider the random graph $G \sim G(n,p)$ with expected degrees $d\leq C$ for some constant $C > 1$. Show that with high probability (say 0.9), all the vertices of $G$ have degree $\mathcal{O}(\tfrac{\log n}{\log\log n})$.
\end{problem}

\begin{solution}
    \vfill
\end{solution}
\clearpage

\end{document}
